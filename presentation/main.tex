% -*- mode: latex; coding: utf-8 -*-
\newcommand{\drawblock}[5]{% PARAMETERS: COLOR, CORNERCOORDS, SIZE
% TOP SIDE
\color{#1!55}
\pgfmoveto{\pgfrelative{\pgfxyz(#2,#3,#4)}{\pgfxyz(0,#5,0)}}
\pgflineto{\pgfrelative{\pgfxyz(#2,#3,#4)}{\pgfxyz(#5,#5,0)}}
\pgflineto{\pgfrelative{\pgfxyz(#2,#3,#4)}{\pgfxyz(#5,#5,#5)}}
\pgflineto{\pgfrelative{\pgfxyz(#2,#3,#4)}{\pgfxyz(0,#5,#5)}}
\pgflineto{\pgfrelative{\pgfxyz(#2,#3,#4)}{\pgfxyz(0,#5,0)}}
\pgffill
\color{black}
\pgfmoveto{\pgfrelative{\pgfxyz(#2,#3,#4)}{\pgfxyz(0,#5,0)}}
\pgflineto{\pgfrelative{\pgfxyz(#2,#3,#4)}{\pgfxyz(#5,#5,0)}}
\pgflineto{\pgfrelative{\pgfxyz(#2,#3,#4)}{\pgfxyz(#5,#5,#5)}}
\pgflineto{\pgfrelative{\pgfxyz(#2,#3,#4)}{\pgfxyz(0,#5,#5)}}
\pgflineto{\pgfrelative{\pgfxyz(#2,#3,#4)}{\pgfxyz(0,#5,0)}}
\pgfstroke
% RIGHT SIDE
\color{#1!65}
\pgfmoveto{\pgfrelative{\pgfxyz(#2,#3,#4)}{\pgfxyz(#5,0,0)}}
\pgflineto{\pgfrelative{\pgfxyz(#2,#3,#4)}{\pgfxyz(#5,#5,0)}}
\pgflineto{\pgfrelative{\pgfxyz(#2,#3,#4)}{\pgfxyz(#5,#5,#5)}}
\pgflineto{\pgfrelative{\pgfxyz(#2,#3,#4)}{\pgfxyz(#5,0,#5)}}
\pgflineto{\pgfrelative{\pgfxyz(#2,#3,#4)}{\pgfxyz(#5,0,0)}}
\pgffill
\color{black}
\pgfmoveto{\pgfrelative{\pgfxyz(#2,#3,#4)}{\pgfxyz(#5,0,0)}}
\pgflineto{\pgfrelative{\pgfxyz(#2,#3,#4)}{\pgfxyz(#5,#5,0)}}
\pgflineto{\pgfrelative{\pgfxyz(#2,#3,#4)}{\pgfxyz(#5,#5,#5)}}
\pgflineto{\pgfrelative{\pgfxyz(#2,#3,#4)}{\pgfxyz(#5,0,#5)}}
\pgflineto{\pgfrelative{\pgfxyz(#2,#3,#4)}{\pgfxyz(#5,0,0)}}
\pgfstroke
%FRONT
\color{#1!99}
\pgfmoveto{\pgfxyz(#2,#3,#4)}
\pgflineto{\pgfrelative{\pgfxyz(#2,#3,#4)}{\pgfxyz(#5,0,0)}}
\pgflineto{\pgfrelative{\pgfxyz(#2,#3,#4)}{\pgfxyz(#5,#5,0)}}
\pgflineto{\pgfrelative{\pgfxyz(#2,#3,#4)}{\pgfxyz(0,#5,0)}}
\pgflineto{\pgfxyz(#2,#3,#4)}
\pgffill
\color{black}
\pgfmoveto{\pgfxyz(#2,#3,#4)}
\pgflineto{\pgfrelative{\pgfxyz(#2,#3,#4)}{\pgfxyz(#5,0,0)}}
\pgflineto{\pgfrelative{\pgfxyz(#2,#3,#4)}{\pgfxyz(#5,#5,0)}}
\pgflineto{\pgfrelative{\pgfxyz(#2,#3,#4)}{\pgfxyz(0,#5,0)}}
\pgflineto{\pgfxyz(#2,#3,#4)}
\pgfstroke
}

\newcommand{\RsGsB}{\begin{pgfpicture}{0.5mm}{0.5mm}{12mm}{5mm}
\pgfsetxvec{\pgfpoint{0.4cm}{0cm}}
\pgfsetyvec{\pgfpoint{0cm}{0.4cm}}
\pgfsetzvec{\pgfpoint{0.15cm}{0.15cm}}
\drawblock{red}{0}{0}{0}{1}
\drawblock{green}{1}{0}{0}{1}
\drawblock{blue}{2}{0}{0}{1}
\end{pgfpicture}
}

\newcommand{\twostacksA}[3]{\begin{pgfpicture}{0.5mm}{0.5mm}{0.8cm}{0.9cm}
\pgfsetxvec{\pgfpoint{0.4cm}{0cm}}
\pgfsetyvec{\pgfpoint{0cm}{0.4cm}}
\pgfsetzvec{\pgfpoint{0.15cm}{0.15cm}}
\drawblock{#2}{0}{0}{0}{1}
\drawblock{#1}{0}{1}{0}{1}
\drawblock{#3}{1}{0}{0}{1}
\end{pgfpicture}
}

\newcommand{\twostacksB}[3]{\begin{pgfpicture}{0.5mm}{0.5mm}{0.8cm}{0.9cm}
\pgfsetxvec{\pgfpoint{0.4cm}{0cm}}
\pgfsetyvec{\pgfpoint{0cm}{0.4cm}}
\pgfsetzvec{\pgfpoint{0.15cm}{0.15cm}}
\drawblock{#1}{0}{0}{0}{1}
\drawblock{#3}{1}{0}{0}{1}
\drawblock{#2}{1}{1}{0}{1}
\end{pgfpicture}
}

\newcommand{\onestack}[3]{\begin{pgfpicture}{0.5mm}{0.5mm}{4mm}{13mm}
\pgfsetxvec{\pgfpoint{0.4cm}{0cm}}
\pgfsetyvec{\pgfpoint{0cm}{0.4cm}}
\pgfsetzvec{\pgfpoint{0.15cm}{0.15cm}}
\drawblock{#3}{0}{0}{0}{1}
\drawblock{#2}{0}{1}{0}{1}
\drawblock{#1}{0}{2}{0}{1}
\end{pgfpicture}
}

\newcommand{\RGsB}{\twostacksA{red}{green}{blue}}

\newcommand{\RBsG}{\twostacksB{green}{red}{blue}}
\newcommand{\RBsGr}{\twostacksA{red}{blue}{green}}

\newcommand{\BRsG}{\twostacksA{blue}{red}{green}}
\newcommand{\BRsGr}{\twostacksB{green}{blue}{red}}

\newcommand{\RsBG}{\twostacksB{red}{blue}{green}}

\newcommand{\GRsB}{\twostacksA{green}{red}{blue}}
\newcommand{\RsGB}{\twostacksB{red}{green}{blue}}

\newcommand{\RGB}{\onestack{red}{green}{blue}}
\newcommand{\RBG}{\onestack{red}{blue}{green}}
\newcommand{\GRB}{\onestack{green}{red}{blue}}
\newcommand{\GBR}{\onestack{green}{blue}{red}}
\newcommand{\BRG}{\onestack{blue}{red}{green}}
\newcommand{\BGR}{\onestack{blue}{green}{red}}

%\documentclass{gkibeamer}
\documentclass{beamer}

\usepackage{tikz}
\usepackage{ifthen}
\usepackage{stmaryrd}
\usepackage[T1]{fontenc}
\usepackage[brazil]{babel}
\usepackage[utf8]{inputenc}
\usetheme{Inf}

\setbeamertemplate{footline}[frame number]

\title[Discovering and Learning Preferred Operators]{Discovering and Learning Preferred Operators for Classical Planning with Neural Networks}
\author{Pedro Probst Minini}

\institute{Federal University of Rio Grande do Sul\\Institute of Informatics\\Department of Theoretical Informatics}
\subject{AI}

\newcommand{\bs}{\texttt{\char`\\}} %% backslash in tt font
\newcommand{\us}{\texttt{\char`_}} %% underscore in tt font

%% Indentation and other things for algorithms
\newcommand{\comment}[1]{\hspace*{\fill}\hilite{[#1]}}
%% for pseudo-code comments
\newcommand{\keyword}[1]{\ensuremath{\textup{\textbf{#1}}}}
\newcommand{\func}[1]{\ensuremath{\textup{#1}}}
%% for pseudo-code
\newcommand{\var}[1]{\ensuremath{\textit{#1}}}
%% for pseudo-code and math variables
\newcommand{\val}[1]{\ensuremath{\textup{#1}}}
%% for values in a finite-domain variable's domain
\newcommand{\op}[1]{\ensuremath{\textit{#1}}}
%% for actions in action sequences in plans etc. (note that \action is
%% already used by LaTeX or some package)

\newcommand{\ind}{{}\qquad}
\newcommand{\indtwo}{\ind\qquad}
\newcommand{\indthree}{\indtwo\qquad}
\newcommand{\indfour}{\indthree\qquad}
\newcommand{\indfive}{\indfour\qquad}

%% Space-saving version of center environment.
\newenvironment{tightcenter}{\centering}{}

%% Math environments
\newenvironment{tightalign}[1][c]{\par\(\begin{array}[#1]{@{}r@{}l}}
               {\end{array}\)\par}
\newenvironment{tightalignnopar}[1][c]{\(\begin{array}[#1]{@{}r@{}l}}
               {\end{array}\)}
\newenvironment{wrappedmath}[1][t]{\begin{array}[#1]{@{}l}}{\end{array}}

%% Don't use proof environment for start of proof since that
%% automatically adds a qed symbol at the end.
\newenvironment{proofstart}{\begin{block}{Proof.}}{\hspace*{\fill}\dots\end{block}}
\newenvironment{proofmiddle}{\begin{block}{Proof (continued).}}{\hspace*{\fill}\dots\end{block}}
\newenvironment{proofend}{\begin{proof}[Proof (continued).]}{\end{proof}}
\newenvironment{proofsketch}{\begin{block}{Proof Sketch.}}{\end{block}}
\newenvironment{proofsketchstart}{\begin{block}{Proof Sketch.}}{\hspace*{\fill}\dots\end{block}}
\newenvironment{proofsketchmiddle}{\begin{block}{Proof Sketch (continued).}}{\hspace*{\fill}\dots\end{block}}
\newenvironment{proofsketchend}{\begin{proof}[Proof Sketch (continued).]}{\end{proof}}
\newcommand{\tbc}{\hspace*{\fill}\dots}

\newtheorem{proposition}{Proposition}

%% Basic text stuff

\newcommand{\hilite}[1]{\textcolor{structure.fg}{#1}}

\newcommand{\ie}{i.e.}
\newcommand{\eg}{e.g.}

\newcommand{\grey}[1]{\textcolor{lightgray}{#1}}
\newcommand{\textred}[1]{{\color{red} #1}}
\newcommand{\textblue}[1]{{\color{blue} #1}}
\newcommand{\textgreen}[1]{{\color{green} #1}}

%% Algorithms, mathematical notation etc.

\newcommand{\hstar}{\ensuremath{h^*}}
\newcommand{\rstar}{\ensuremath{r^*}}
\newcommand{\astar}{\ensuremath{\textup{A}^*}}
\newcommand{\idastar}{\ensuremath{\textup{IDA}^*}}
\newcommand{\dom}{\textup{dom}}
\newcommand{\sasplus}{\ensuremath{\textup{SAS}^+}}

\newcommand{\true}{\ensuremath{\mathbf{T}}}
\newcommand{\false}{\ensuremath{\mathbf{F}}}

\newcommand{\pre}{\ensuremath{\textit{pre}}}
\newcommand{\eff}{\ensuremath{\textit{eff}}}
\newcommand{\cost}{\ensuremath{\textit{cost}}}
\newcommand{\add}{\ensuremath{\textit{add}}}
\newcommand{\del}{\ensuremath{\textit{del}}}
\newcommand{\precond}{\ensuremath{\textit{prec}}}

\newcommand{\applyop}[2]{#2\llbracket#1\rrbracket}
\newcommand{\applyplan}[2]{#2\llbracket#1\rrbracket}
%% These macros should no longer be used.
%% TODO: Remove them entirely rather than commenting them out once
%% we're happy with our revisions of the parts that used to use them.
%% \newcommand{\changes}[2]{\lbrack #1\rbrack_{#2}}
%% \newcommand{\addchanges}[2]{\lbrack #1\rbrack^+_{#2}}
%% \newcommand{\delchanges}[2]{\lbrack #1\rbrack^-_{#2}}
\newcommand{\condeff}{\vartriangleright}

\newcommand{\addset}{\ensuremath{\textit{addset}}}
\newcommand{\delset}{\ensuremath{\textit{delset}}}

\newcommand{\effcond}{\ensuremath{\textit{effcond}}}
\newcommand{\consist}{\ensuremath{\textit{consist}}}

\newcommand{\vars}{\ensuremath{\textit{vars}}}

\newcommand{\regr}{\ensuremath{\textit{regr}}}
\newcommand{\regrstrips}{\ensuremath{\textit{sregr}}}
\newcommand{\eprecon}[2]{\textit{EPC}_{#1}(#2)}
\newcommand{\sregrpairs}[2]{R(#2, #1)}
\newcommand{\varset}[1]{\ensuremath{\textit{vars}(#1)}}
\newcommand{\conj}[1]{\ensuremath{\textit{conj}(#1)}}

%% Blocks world examples
\newcommand{\AONB}{\var{A-on-B}}
\newcommand{\AONC}{\var{A-on-C}}
\newcommand{\AOND}{\var{A-on-D}}
\newcommand{\BONA}{\var{B-on-A}}
\newcommand{\BONC}{\var{B-on-C}}
\newcommand{\BOND}{\var{B-on-D}}
\newcommand{\CONA}{\var{C-on-A}}
\newcommand{\CONB}{\var{C-on-B}}
\newcommand{\COND}{\var{C-on-D}}
\newcommand{\DONA}{\var{D-on-A}}
\newcommand{\DONB}{\var{D-on-B}}
\newcommand{\DONC}{\var{D-on-C}}
\newcommand{\AONTABLE}{\var{A-on-table}}
\newcommand{\BONTABLE}{\var{B-on-table}}
\newcommand{\CONTABLE}{\var{C-on-table}}
\newcommand{\CLEARA}{\var{A-clear}}
\newcommand{\CLEARB}{\var{B-clear}}
\newcommand{\CLEARC}{\var{C-clear}}

%% Macros to align the width of things.
\newlength{\mywidth}
\newcommand{\setmywidth}[1]{\settowidth{\mywidth}{#1}}
\newcommand{\usemywidth}[1]{\makebox[\mywidth][l]{#1}}
\newcommand{\usemywidthmath}[1]{\usemywidth{\ensuremath{#1}}}

\newcounter{mysaveenumi}
\newcommand{\enumtbc}{\setcounter{mysaveenumi}{\theenumi}}
\newcommand{\continueenum}{\setcounter{enumi}{\themysaveenumi}}

%% Complexity classes.
\newcommand{\decisionclass}[1]{\ensuremath{\textsf{\textup{#1}}}}
\newcommand{\dtime}{\decisionclass{DTIME}}
\newcommand{\ntime}{\decisionclass{NTIME}}
\newcommand{\dspace}{\decisionclass{DSPACE}}
\newcommand{\nspace}{\decisionclass{NSPACE}}
\newcommand{\ptime}{\decisionclass{P}}
\newcommand{\np}{\decisionclass{NP}}
\newcommand{\pspace}{\decisionclass{PSPACE}}
\newcommand{\npspace}{\decisionclass{NPSPACE}}
\newcommand{\exptime}{\decisionclass{EXP}}
\newcommand{\expspace}{\decisionclass{EXPSPACE}}
\newcommand{\dblexptime}{\decisionclass{2-EXP}}
\newcommand{\dblexpspace}{\decisionclass{2-EXPSPACE}}

%% Turing machine stuff.
\newcommand{\accept}{{\textsf{Y}}}

%% Decision problems and related things.
\newcommand{\planex}{\textsc{PlanEx}}
\newcommand{\bcplanex}{\textsc{BCPlanEx}}
\newcommand{\easier}{\ensuremath{\le_{\text{p}}}}

%% Various stuff.
\newcommand{\relaxation}[1]{#1^+}
\newcommand{\onset}[1]{\textit{on}(#1)}

%% Heuristics.
\newcommand{\hplus}{\ensuremath{h^+}}
\newcommand{\hmax}{\ensuremath{h^{\text{max}}}}
\newcommand{\hadd}{\ensuremath{h^{\text{add}}}}
\newcommand{\hlmcut}{\ensuremath{h^{\text{LM-cut}}}}
\newcommand{\hff}{\ensuremath{h^{\text{FF}}}}
\newcommand{\hcs}{\ensuremath{h^{\text{cs}}}}
\newcommand{\hsa}{\ensuremath{h^{\text{sa}}}}
\newcommand{\hlst}{\ensuremath{h^{\text{lst}}}}
\newcommand{\hm}{\ensuremath{h^m}}
\newcommand{\hmhs}{\ensuremath{h^\text{MHS}}}
\newcommand{\hucp}{\ensuremath{h^\text{UCP}}}
\newcommand{\hlmcount}{\ensuremath{h^\text{LM-count}}}
\newcommand{\hflow}{\ensuremath{h^\text{flow}}}
\newcommand{\hposthoc}[1][]{\ensuremath{h_{#1}^{\textup{PhO}}}}
\newcommand{\hcanon}{\ensuremath{h^{\mathcal{C}}}}
\newcommand{\hocp}{\ensuremath{h^\textup{OCP}}}

%% Used for AND/OR graphs.
%% Note that \succ is already used by LaTeX.
\newcommand{\suc}{\ensuremath{\textit{succ}}}

%% Used for abstraction chapters.
\newcommand{\graphequiv}{\stackrel{\textup{G}}{\sim}}
\newcommand{\cg}{\ensuremath{\textit{CG}}}
\newcommand{\hhhh}{\ensuremath{h_{\text{HHH}}}}

\newcommand{\cliques}{\ensuremath{\textit{cliques}}}

% AND/OR landmarks
\newcommand{\andorarc}[2]{\ensuremath{\langle #1,#2\rangle}}
\newcommand{\landmark}{\textit{LM}}

% Landmark orderings
\newcommand{\naturalord}[2]{\ensuremath{#1\rightarrow #2}}
\newcommand{\necessaryord}[2]{\ensuremath{#1\rightarrow_{\textup{n}} #2}}
\newcommand{\greedynecessaryord}[2]{\ensuremath{#1\rightarrow_{\textup{gn}} #2}}
\newcommand{\arbitraryord}[2]{\ensuremath{#1\rightarrow_{\textup{x}} #2}}


\newcommand{\lpvar}[1]{\ensuremath{\hilite{#1}}}
\newcommand{\ocvar}[1]{\lpvar{\textup{Count}_{#1}}}

\providecommand{\floor}[1]{\ensuremath{\left\lfloor #1\right\rfloor}}
\providecommand{\ceil}[1]{\ensuremath{\left\lceil #1\right\rceil}}


% #############################################################################################################
%
%% Needed for "Content of this Course" slide in most chapters:
\usetikzlibrary{positioning, arrows.meta, patterns, automata}

\usepackage[linesnumbered, ruled, vlined]{algorithm2e}
\usepackage{mathtools}
\usepackage{booktabs}
\usepackage{comment}
%\usepackage{natbib}
\usepackage[citestyle=authoryear,maxnames=10,maxcitenames=1,giveninits=true,backend=biber,uniquelist=false]{biblatex}
\usepackage[utf8]{inputenc}
\usepackage[textsize=tiny,colorinlistoftodos,prependcaption]{todonotes}
\addbibresource{biblio.bib}

\newcommand{\pp}[2][noinline]{\todo[color=purple!50,linecolor={purple!100},#1,fancyline,author=Pedro]{#2}}
\newcommand{\ppi}[2][inline]{\todo[color=purple!50,linecolor={purple!100},#1,fancyline,author=Pedro]{#2}}

% #####################
\AtBeginSubsection[]
{
\begin{frame}[noframenumbering]
    \frametitle{\n}
    \tableofcontents[currentsection,currentsubsection]
\end{frame}
}
% #####################

% #############################################################################################################

%\subtitle{Some title}
%\date{}
%\begin{document}

%\begin{frame}{Outline}
%\tableofcontents
%\end{frame}

% Define the title with \title[short title]{long title}
% Short title is optional
\title[Discovering and Learning Preferred Operators]
      {Discovering and Learning Preferred Operators for Classical Planning with Neural Networks}

% Optional subtitle
%\subtitle{Defesa de Mestrado}

\date{Julho de 2023}

% Author information
\author{Pedro Probst Minini}
\institute{Instituto de Informática --- UFRGS\\\texttt{inf.ufrgs.br/\~{}bmenegola}}

\begin{document}

% Command to create title page
\InfTitlePage

\begin{frame}{Outline}
  \frametitle{Agenda}
  \tableofcontents
\end{frame}

\section{Introdução}
\subsection{Planejamento}
\begin{frame}{Planejamento Clássico}
\emph{Planejamento} tem o objetivo de encontrar uma \alert{sequência de ações} a partir de um \alert{estado inicial} que satisfaça as \alert{condições objetivo}.
    \begin{exampleblock}{\strut Exemplo: Planejamento Clássico}
      \hilite{ambiente}
      \begin{itemize}
      \item \alert{estático} vs.\ \grey{dinâmico}
        \pause
      \item \alert{determinístico} vs.\ \grey{não determinístico}
        vs.\ \grey{estocástico}
        \pause
      \item \alert{observável}
        vs.\ \grey{parcialmente observável}
        vs.\ \grey{não observável}
        \pause
      \item \alert{discreto} vs.\ \grey{contínuo}
        \pause
      \item \alert{agente único} vs.\ \grey{agentes multiplos}
      \end{itemize}

      \pause
      \hilite{método de resolução}
      \begin{itemize}
      \item \grey{específico por problema} vs.\ \alert{geral} vs.\ \orange{aprendizado}
      \end{itemize}
    \end{exampleblock}
% ppi{Then briefly talk about non-deterministic planning.}
% Classical Planning is an example of a "model".
\end{frame}

\begin{frame}{STRIPS}
  \begin{definition}[Um problema de planejamento em STRIPS]
    $\Pi = \langle F, O, s_{0}, s^{*}\rangle$ onde
    \begin{itemize}
        \item $F$ é um conjunto de variáveis booleanas (\alert{fatos ou proposições}),
        \item $O$ é um conjunto de \alert{operadores} ou ações sobre $F$, onde $\langle pre(o), add(o), del(o) \rangle \subseteq F$,
        \item $s_{0} \subseteq F$ é o conjunto de fatos que representa o \alert{estado inicial}, e
        \item $s^{*} \subseteq F$ é o conjunto de fatos que devem ser satisfeitos (``\alert{gol}'').
    \end{itemize}
   \pause
    As ações $A(s)$ são \alert{aplicáveis} em $s$ se elas satisfizerem $pre(o)$.

    Avançamos de um estado $s$ com o operador $o$ definindo as proposições em $add(o)$ como \alert{verdadeiras} e em $del(o)$ como \alert{falsas}.
    \pause

    Por fim, $\pi = o_{1}, o_{2},\ldots, o_{n}$ é chamado de \alert{plano} para $\Pi$, em que $o_{i}$ é um operador aplicável.
\end{definition}
%Planning tasks are usually formally described in PDDL files, following a Lisp-like structure.
\end{frame}

\begin{frame}{Example: Blocksworld} 
\begin{figure}
    \centering
    \includegraphics[width=7cm]{img/blocksworld1.png}
    \caption{Uma tarefa de planejamento do domínio Blocks World.}
\end{figure}
\end{frame}

\begin{frame}{Exemplo: Blocksworld}
% This is from Malte Helmert's slides, slightly modified.
    \begin{exampleblock}{\strut Exemplo: Blocks World}
      \hilite{$\Pi = \langle F, O, s_{0}, s^{*}\rangle$}
      \begin{itemize}
      \item $F$ = \{on(a,b), on(a,c), on(b,a), on(b,c), on(c,a), on(c,b), on-table(a), on-table(b), on-table(c), clear(a), clear(b), clear(c)\}

       \item $O$ = \{move(a,b,c), move(a,c,b), move(b,a,c), move(b,c,a), move(c,a,b), move(c,b,a), to-table(a,b), to-table(a,c), to-table(b,a), to-table(b,c), to-table(c,a), to-table(c,b), from-table(a,b), from-table(a,c), from-table(b,a), from-table(b,c), from-table(c,a), from-table(c,b)\}

        \item $s_{0}$ = \{on(c,a), on-table(a), on-table(b), clear(c), clear(b)\}

         \item $s^{*}$ = \{on(a,b), on(b,c)\}
      \end{itemize}
    \end{exampleblock}
\end{frame}

\begin{frame}{Exemplo: Blocks World}
    \begin{exampleblock}{\strut Exemmplo: Blocks World}
      \begin{itemize}
      \item \emph{move}: move a block from one block to another.
      \begin{itemize}
        \item Pre(move(a,b,c)) = \{on(a,b), clear(a), clear(c)\}
        \item Add(move(a,b,c)) = \{on(a,c), clear(b)\}
        \item Del(move(a,b,c)) = \{on(a,b), clear(c)\}
        \item Com base no estado inicial, essa ação é aplicável? % Não, ela não satisfaz as condições prévias.
      \end{itemize}
      \item \emph{to-table}: mover um bloco de um bloco para a mesa.
      \item \emph{from-table}: mover um bloco de um bloco para a mesa.
      \end{itemize}
    \end{exampleblock}
\end{frame}

\begin{frame}{Espaço de estados do Blocks World com 3 blocos}
  \begin{center}
    \begin{pgfpicture}{-39mm}{-32.4mm}{39mm}{32.4mm}
      \pgfnodebox{RsGsB}[virtual]{\pgfpolar{0}{0cm}}{\RsGsB}{2pt}{2pt}

      \pgfnodebox{RGsB}[virtual]{\pgfpolar{0}{16mm}}{\RGsB}{2pt}{2pt}
      \pgfnodebox{RBsG}[virtual]{\pgfpolar{60}{16mm}}{\RBsG}{2pt}{2pt}
      \pgfnodebox{BRsG}[virtual]{\pgfpolar{120}{16mm}}{\BRsG}{2pt}{2pt}
      \pgfnodebox{RsBG}[virtual]{\pgfpolar{180}{16mm}}{\RsBG}{2pt}{2pt}
      \pgfnodebox{RsGB}[virtual]{\pgfpolar{240}{16mm}}{\RsGB}{2pt}{2pt}
      \pgfnodebox{GRsB}[virtual]{\pgfpolar{300}{16mm}}{\GRsB}{2pt}{2pt}

      \pgfnodebox{BRG}[virtual]{\pgfpolar{0}{32mm}}{\BRG}{2pt}{2pt}
      \pgfnodebox{GRB}[virtual]{\pgfpolar{50}{32mm}}{\GRB}{2pt}{2pt}
      \pgfnodebox{GBR}[virtual]{\pgfpolar{130}{32mm}}{\GBR}{2pt}{2pt}
      \pgfnodebox{RBG}[virtual]{\pgfpolar{180}{32mm}}{\RBG}{2pt}{2pt}
      \pgfnodebox{RGB}[virtual]{\pgfpolar{230}{32mm}}{\RGB}{2pt}{2pt}
      \pgfnodebox{BGR}[virtual]{\pgfpolar{310}{32mm}}{\BGR}{2pt}{2pt}

      \pgfsetendarrow{\pgfarrowtriangle{5pt}}
      \pgfsetstartarrow{\pgfarrowtriangle{5pt}}

      \pgfnodeconnline{RsGsB}{RGsB}
      \pgfnodeconnline{RsGsB}{RBsG}
      \pgfnodeconnline{RsGsB}{BRsG}
      \pgfnodeconnline{RsGsB}{RsBG}
      \pgfnodeconnline{RsGsB}{RsGB}
      \pgfnodeconnline{RsGsB}{GRsB}

      \pgfnodeconnline{RGsB}{BRG}
      \pgfnodeconnline{RBsG}{GRB}
      \pgfnodeconnline{BRsG}{GBR}
      \pgfnodeconnline{RsBG}{RBG}
      \pgfnodeconnline{RsGB}{RGB}
      \pgfnodeconnline{GRsB}{BGR}

      \pgfnodeconnline{RGsB}{RBsG}
      \pgfnodeconnline{BRsG}{RsBG}
      \pgfnodeconnline{RsGB}{GRsB}
    \end{pgfpicture}
  \end{center}
\end{frame}

\begin{frame}{\sas}
\begin{itemize}
\item \sas é similar a STRIPS, mas variáveis têm domínios finitos, não sendo necessariamente binárias.
  \item O estado inicial $s^{*}$ é tipicamente um \alert{estado completo}, i.e., todas as variáveis estão definidas.
  \item O gol $s^{*}$ é definido \alert{parcialmente}, com certas variáveis indefinidas ($\bot$).
\end{itemize}
\end{frame}

\subsection{Busca Heurística}
\begin{frame}{Busca heurística}
\begin{itemize}
  \item Planejadores tipicamente buscam satisfazer o gol \alert{priorizando} estados no espaço de estados com \alert{menor função heurística}.
  \pause
  \item Um estado $s$ tem uma função heurística $h(s)$, dado por uma função heurística $h$, que estima o \alert{custo de atingir o gol $s^{*}$} a partir de $s$.
  \pause
  \item Algoritmos ``best-first search'', e.g., greedy best-first serach (GBFS), \astar.
\end{itemize}
\end{frame}

\subsection{Operadores Preferidos}
\begin{frame}{Operadores preferidos (PO)}
\begin{itemize}
\item \alert{Operadores} considerados \alert{promissores} ``por alguma razão''.
    \pause
  \begin{itemize}
  \item A definição não é fechada, mas tipicamente isso quer dizer ``operadores que levam a algum estado mais provável de satisfazer o gol''. % Helmert e eu.
  \end{itemize}
  \pause
\item Usados em conjunto com funções heurísticas.
  \begin{itemize}
  \item Sozinhos, são equivalentes a políticas (policies).
  \end{itemize}
\pause
\item No Fast Downward: usados em esquema ``\alert{dual-queue}''.
  \begin{itemize}
  \item Uma fila para todos os estados, outra para \alert{estados gerados exclusivamente via POs}.
  \item Expansão alternada, ou não (\alert{boosting}).
  \end{itemize}
\end{itemize}
\end{frame}

\begin{frame}{Operadores preferidos do FF}
\end{frame}

\subsection{Aprendendo Funções Heurísticas}
\begin{frame}{Aprendizado de funções heurísticas}
\end{frame}

\section{Abordagem proposta}
\begin{frame}{Como aprender operadores preferidos?}
\end{frame}

\subsection{Operadores Preferidos Ideais}
\begin{frame}{Operadores preferidos ideais}
\end{frame}

\subsection{Operadores Preferidos Descobertos}
\begin{frame}{Operadores preferidos descobertos}
\end{frame}

\subsection{Gerando Amostras com \bfsrs}
\begin{frame}{Gerando amostras com \bfsrs}
\end{frame}

\section{Experimentos}
\begin{frame}{Configuração}
\end{frame}

\subsection{Aprendendo Operadores Preferidos}
\begin{frame}{}
\end{frame}

%\subsection{Conjuntos de Amostras de Diferentes Tamanhos}
%\begin{frame}{}
%\end{frame}

\subsection{\bfsrs vs. \bfsrw}
\begin{frame}{}
\end{frame}

\subsection{Operadores Preferidos com Funções Heurísticas Lógicas}
\begin{frame}{}
\end{frame}


\section{Conclusão}
\begin{frame}{Conclusão}
\end{frame}

\printbibliography

\end{document}
