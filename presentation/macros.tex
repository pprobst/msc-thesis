\newcommand{\bs}{\texttt{\char`\\}} %% backslash in tt font
\newcommand{\us}{\texttt{\char`_}} %% underscore in tt font

%% Indentation and other things for algorithms
\newcommand{\comment}[1]{\hspace*{\fill}\hilite{[#1]}}
%% for pseudo-code comments
\newcommand{\keyword}[1]{\ensuremath{\textup{\textbf{#1}}}}
\newcommand{\func}[1]{\ensuremath{\textup{#1}}}
%% for pseudo-code
\newcommand{\var}[1]{\ensuremath{\textit{#1}}}
%% for pseudo-code and math variables
\newcommand{\val}[1]{\ensuremath{\textup{#1}}}
%% for values in a finite-domain variable's domain
\newcommand{\op}[1]{\ensuremath{\textit{#1}}}
%% for actions in action sequences in plans etc. (note that \action is
%% already used by LaTeX or some package)

\newcommand{\ind}{{}\qquad}
\newcommand{\indtwo}{\ind\qquad}
\newcommand{\indthree}{\indtwo\qquad}
\newcommand{\indfour}{\indthree\qquad}
\newcommand{\indfive}{\indfour\qquad}

%% Space-saving version of center environment.
\newenvironment{tightcenter}{\centering}{}

%% Math environments
\newenvironment{tightalign}[1][c]{\par\(\begin{array}[#1]{@{}r@{}l}}
               {\end{array}\)\par}
\newenvironment{tightalignnopar}[1][c]{\(\begin{array}[#1]{@{}r@{}l}}
               {\end{array}\)}
\newenvironment{wrappedmath}[1][t]{\begin{array}[#1]{@{}l}}{\end{array}}

%% Don't use proof environment for start of proof since that
%% automatically adds a qed symbol at the end.
\newenvironment{proofstart}{\begin{block}{Proof.}}{\hspace*{\fill}\dots\end{block}}
\newenvironment{proofmiddle}{\begin{block}{Proof (continued).}}{\hspace*{\fill}\dots\end{block}}
\newenvironment{proofend}{\begin{proof}[Proof (continued).]}{\end{proof}}
\newenvironment{proofsketch}{\begin{block}{Proof Sketch.}}{\end{block}}
\newenvironment{proofsketchstart}{\begin{block}{Proof Sketch.}}{\hspace*{\fill}\dots\end{block}}
\newenvironment{proofsketchmiddle}{\begin{block}{Proof Sketch (continued).}}{\hspace*{\fill}\dots\end{block}}
\newenvironment{proofsketchend}{\begin{proof}[Proof Sketch (continued).]}{\end{proof}}
\newcommand{\tbc}{\hspace*{\fill}\dots}

\newtheorem{proposition}{Proposition}

%% Basic text stuff

\newcommand{\hilite}[1]{\textcolor{structure.fg}{#1}}

\newcommand{\ie}{i.e.}
\newcommand{\eg}{e.g.}

\newcommand{\grey}[1]{\textcolor{lightgray}{#1}}
\newcommand{\textred}[1]{{\color{red} #1}}
\newcommand{\orange}[1]{{\color{orange} #1}}
\newcommand{\textblue}[1]{{\color{blue} #1}}
\definecolor{green}{HTML}{28cd4c}
\newcommand{\textgreen}[1]{{\color{green} #1}}
\newcommand{\textorange}[1]{{\color{orange} #1}}

%% Algorithms, mathematical notation etc.

\newcommand{\hstar}{\ensuremath{h^*}}
\newcommand{\rstar}{\ensuremath{r^*}}
\newcommand{\astar}{\ensuremath{\textup{A}^*}}
\newcommand{\idastar}{\ensuremath{\textup{IDA}^*}}
\newcommand{\dom}{\textup{dom}}
\newcommand{\sasplus}{\ensuremath{\textup{SAS}^+}}

\newcommand{\true}{\ensuremath{\mathbf{T}}}
\newcommand{\false}{\ensuremath{\mathbf{F}}}

%\newcommand{\pre}{\ensuremath{\textit{pre}}}
%\newcommand{\eff}{\ensuremath{\textit{eff}}}
\newcommand{\cost}{\ensuremath{\textit{cost}}}
%\newcommand{\add}{\ensuremath{\textit{add}}}
%\newcommand{\del}{\ensuremath{\textit{del}}}
%\newcommand{\precond}{\ensuremath{\textit{prec}}}

\newcommand{\applyop}[2]{#2\llbracket#1\rrbracket}
\newcommand{\applyplan}[2]{#2\llbracket#1\rrbracket}
%% These macros should no longer be used.
%% TODO: Remove them entirely rather than commenting them out once
%% we're happy with our revisions of the parts that used to use them.
%% \newcommand{\changes}[2]{\lbrack #1\rbrack_{#2}}
%% \newcommand{\addchanges}[2]{\lbrack #1\rbrack^+_{#2}}
%% \newcommand{\delchanges}[2]{\lbrack #1\rbrack^-_{#2}}
\newcommand{\condeff}{\vartriangleright}

\newcommand{\addset}{\ensuremath{\textit{addset}}}
\newcommand{\delset}{\ensuremath{\textit{delset}}}

\newcommand{\effcond}{\ensuremath{\textit{effcond}}}
\newcommand{\consist}{\ensuremath{\textit{consist}}}

\newcommand{\vars}{\ensuremath{\textit{vars}}}

\newcommand{\regr}{\ensuremath{\textit{regr}}}
\newcommand{\regrstrips}{\ensuremath{\textit{sregr}}}
\newcommand{\eprecon}[2]{\textit{EPC}_{#1}(#2)}
\newcommand{\sregrpairs}[2]{R(#2, #1)}
\newcommand{\varset}[1]{\ensuremath{\textit{vars}(#1)}}
\newcommand{\conj}[1]{\ensuremath{\textit{conj}(#1)}}

%% Blocks world examples
\newcommand{\AONB}{\var{A-on-B}}
\newcommand{\AONC}{\var{A-on-C}}
\newcommand{\AOND}{\var{A-on-D}}
\newcommand{\BONA}{\var{B-on-A}}
\newcommand{\BONC}{\var{B-on-C}}
\newcommand{\BOND}{\var{B-on-D}}
\newcommand{\CONA}{\var{C-on-A}}
\newcommand{\CONB}{\var{C-on-B}}
\newcommand{\COND}{\var{C-on-D}}
\newcommand{\DONA}{\var{D-on-A}}
\newcommand{\DONB}{\var{D-on-B}}
\newcommand{\DONC}{\var{D-on-C}}
\newcommand{\AONTABLE}{\var{A-on-table}}
\newcommand{\BONTABLE}{\var{B-on-table}}
\newcommand{\CONTABLE}{\var{C-on-table}}
\newcommand{\CLEARA}{\var{A-clear}}
\newcommand{\CLEARB}{\var{B-clear}}
\newcommand{\CLEARC}{\var{C-clear}}

%% Macros to align the width of things.
\newlength{\mywidth}
\newcommand{\setmywidth}[1]{\settowidth{\mywidth}{#1}}
\newcommand{\usemywidth}[1]{\makebox[\mywidth][l]{#1}}
\newcommand{\usemywidthmath}[1]{\usemywidth{\ensuremath{#1}}}

\newcounter{mysaveenumi}
\newcommand{\enumtbc}{\setcounter{mysaveenumi}{\theenumi}}
\newcommand{\continueenum}{\setcounter{enumi}{\themysaveenumi}}

%% Complexity classes.
\newcommand{\decisionclass}[1]{\ensuremath{\textsf{\textup{#1}}}}
\newcommand{\dtime}{\decisionclass{DTIME}}
\newcommand{\ntime}{\decisionclass{NTIME}}
\newcommand{\dspace}{\decisionclass{DSPACE}}
\newcommand{\nspace}{\decisionclass{NSPACE}}
\newcommand{\ptime}{\decisionclass{P}}
\newcommand{\np}{\decisionclass{NP}}
\newcommand{\pspace}{\decisionclass{PSPACE}}
\newcommand{\npspace}{\decisionclass{NPSPACE}}
\newcommand{\exptime}{\decisionclass{EXP}}
\newcommand{\expspace}{\decisionclass{EXPSPACE}}
\newcommand{\dblexptime}{\decisionclass{2-EXP}}
\newcommand{\dblexpspace}{\decisionclass{2-EXPSPACE}}

%% Turing machine stuff.
\newcommand{\accept}{{\textsf{Y}}}

%% Decision problems and related things.
\newcommand{\planex}{\textsc{PlanEx}}
\newcommand{\bcplanex}{\textsc{BCPlanEx}}
\newcommand{\easier}{\ensuremath{\le_{\text{p}}}}

%% Various stuff.
\newcommand{\relaxation}[1]{#1^+}
\newcommand{\onset}[1]{\textit{on}(#1)}

%% Heuristics.
\newcommand{\hplus}{\ensuremath{h^+}}
\newcommand{\hmax}{\ensuremath{h^{\text{max}}}}
\newcommand{\hadd}{\ensuremath{h^{\text{add}}}}
\newcommand{\hlmcut}{\ensuremath{h^{\text{LM-cut}}}}
\newcommand{\hff}{\ensuremath{h^{\text{FF}}}}
\newcommand{\hcs}{\ensuremath{h^{\text{cs}}}}
\newcommand{\hsa}{\ensuremath{h^{\text{sa}}}}
\newcommand{\hlst}{\ensuremath{h^{\text{lst}}}}
\newcommand{\hm}{\ensuremath{h^m}}
\newcommand{\hmhs}{\ensuremath{h^\text{MHS}}}
\newcommand{\hucp}{\ensuremath{h^\text{UCP}}}
\newcommand{\hlmcount}{\ensuremath{h^\text{LM-count}}}
\newcommand{\hflow}{\ensuremath{h^\text{flow}}}
\newcommand{\hposthoc}[1][]{\ensuremath{h_{#1}^{\textup{PhO}}}}
\newcommand{\hcanon}{\ensuremath{h^{\mathcal{C}}}}
\newcommand{\hocp}{\ensuremath{h^\textup{OCP}}}

%% Used for AND/OR graphs.
%% Note that \succ is already used by LaTeX.
\newcommand{\suc}{\ensuremath{\textit{succ}}}

%% Used for abstraction chapters.
\newcommand{\graphequiv}{\stackrel{\textup{G}}{\sim}}
\newcommand{\cg}{\ensuremath{\textit{CG}}}
\newcommand{\hhhh}{\ensuremath{h_{\text{HHH}}}}

\newcommand{\cliques}{\ensuremath{\textit{cliques}}}

% AND/OR landmarks
\newcommand{\andorarc}[2]{\ensuremath{\langle #1,#2\rangle}}
\newcommand{\landmark}{\textit{LM}}

% Landmark orderings
\newcommand{\naturalord}[2]{\ensuremath{#1\rightarrow #2}}
\newcommand{\necessaryord}[2]{\ensuremath{#1\rightarrow_{\textup{n}} #2}}
\newcommand{\greedynecessaryord}[2]{\ensuremath{#1\rightarrow_{\textup{gn}} #2}}
\newcommand{\arbitraryord}[2]{\ensuremath{#1\rightarrow_{\textup{x}} #2}}


\newcommand{\lpvar}[1]{\ensuremath{\hilite{#1}}}
\newcommand{\ocvar}[1]{\lpvar{\textup{Count}_{#1}}}

\providecommand{\floor}[1]{\ensuremath{\left\lfloor #1\right\rfloor}}
\providecommand{\ceil}[1]{\ensuremath{\left\lceil #1\right\rceil}}


\providecommand{\multiset}[1]{\ensuremath{\{\!\!\{#1\}\!\!\}}}
\providecommand{\sas}{\ensuremath{\text{SAS}^{+}}\xspace}
\providecommand{\strips}{STRIPS\xspace}
\providecommand{\astar}{\ensuremath{\text{A}^{*}}\xspace}
\providecommand{\gbfs}{\ensuremath{\text{GBFS}}\xspace}
\providecommand{\wida}{\ensuremath{\text{W-IDA}^{*}}\xspace}
\providecommand{\ida}{\ensuremath{\text{IDA}^{*}}\xspace}
\providecommand{\hvalue}[1]{\ensuremath{h^{#1}}\xspace}
\providecommand{\hstarp}[1]{\ensuremath{h^*({#1})}\xspace}
\providecommand{\hff}{\hvalue{\text{FF}}}
\providecommand{\hgc}{\hvalue{\text{GC}}}
\providecommand{\hstar}{\hvalue{*}}
\providecommand{\hlmc}{\hvalue{\text{lm-c}}}
\providecommand{\hmax}{\hvalue{\text{max}}}
\providecommand{\hadd}{\hvalue{\text{add}}}
\providecommand{\hblind}{\hvalue{\text{blind}}}
\providecommand{\h}{$h$\xspace}
\providecommand{\hnn}{$\hat h$\xspace}
\providecommand{\hnrsl}{$\hat h^{\text{N-RSL}}$\xspace}
\providecommand{\hboot}{$\hat h^{\text{Boot}}$\xspace}
\providecommand{\hgc}{\hvalue{\text{gc}}}
\providecommand{\hhgn}{\hvalue{\text{HGN}}}
\providecommand{\unit}{/1\xspace}
\providecommand{\sui}{\text{SUI}\xspace}
\providecommand{\sai}{\text{SAI}\xspace}
\providecommand{\rw}{{RW}\xspace}
\providecommand{\bfs}{{BFS}\xspace}
\providecommand{\dfs}{{DFS}\xspace}
\providecommand{\bfsrw}{\text{FSM}\xspace}
\providecommand{\bfsrs}{\text{XRS}\xspace}
\providecommand{\nn}{{NN}\xspace}
\providecommand{\spp}{{\text{SP}}\xspace}
\providecommand{\fsp}{{\text{FSP}}\xspace}
\providecommand{\bsp}{{\text{BSP}}\xspace}
\providecommand{\hnnrs}{$\hat h{^{20\%}_\text{\meanfx}}$\xspace}
\providecommand{\hnnrsfifty}{$\hat h{^{50\%}_\text{\meanfx}}$\xspace}
\providecommand{\hffexp}{$h^{FF}_{exp}$}
\providecommand{\hgcexp}{$h^{GC}_{exp}$}
\providecommand{\hnnbase}{$\hat h_{0}$\xspace}
\providecommand{\hnnbfs}{$\hat h_{\text{bfs}}$\xspace}
\providecommand{\hnndfs}{$\hat h_{\text{dfs}}$\xspace}
\providecommand{\hnnrw}{$\hat h_{\text{rw}}$\xspace}
\providecommand{\hnnbfsrw}{$\hat h_\text{fsm}$\xspace}
\providecommand{\hnnbfsrwl}[1]{\ensuremath{\hat h_{#1}}\xspace}
\providecommand{\hnnnomutex}{\ensuremath{\hat h^{'}}\xspace}
\providecommand{\hnnnomutexl}[1]{\ensuremath{\hat h^{'}_{#1}}\xspace}
\providecommand{\hnnrsp}[1]{\ensuremath{\hat h_\text{fsm}/^{#1\%}_{\text{RS}}}\xspace}
\providecommand{\hnnrslp}[2]{\ensuremath{\hat h_\text{fsm}^{#1}/^{#2\%}_{\text{RS}}}\xspace}
\providecommand{\define}[1]{#1}
\providecommand{\facts}{\ensuremath{L_F}\xspace}
\providecommand{\meanfx}{\ensuremath{L_{\overline{F}}}\xspace}
\providecommand{\default}{\ensuremath{L_{200}}\xspace}
\providecommand{\distfarthest}{\ensuremath{d^*}\xspace}
\providecommand{\po}[1]{\ensuremath{\hat po^{#1}}\xspace}
\providecommand{\pot}[1]{\ensuremath{po^{#1}}\xspace}
\providecommand{\pog}{\po{\text{G}}}
\providecommand{\pofsm}{\po{\text{FSM}}}
\providecommand{\pogthresh}{\po{\text{G-thresh}}}
\providecommand{\pogmax}{\po{\text{G-max}}}
\providecommand{\popfa}{\po{\text{PFA}}}
\providecommand{\popfo}{\po{\text{PFO}}}
\providecommand{\postar}{\po{*}}
\providecommand{\postartable}{\pot{*}}
\providecommand{\pogstar}{\po{\text{G}^*}}
\providecommand{\pogstarthresh}{\po{\text{OPT-thresh}}}
\providecommand{\pogstarmax}{\po{\text{OPT-max}}}
\providecommand{\popf}{\po{\text{PF}}}
\providecommand{\poff}{\pot{\text{FF}}}
\providecommand{\pogc}{\po{\text{GC}}}

%% mathematical definitions
\ifcsname dom\endcsname\else\DeclareMathOperator{\dom}{dom}\fi
\DeclareMathOperator{\pre}{pre}
\DeclareMathOperator{\eff}{eff}
\DeclareMathOperator{\sucs}{succ}
\DeclareMathOperator{\pred}{pred}
\DeclareMathOperator{\functioninitial}{initial\_state}
\DeclareMathOperator{\functiongoal}{goal\_condition}
\DeclareMathOperator{\mutex}{mutex}
\DeclareMathOperator{\del}{del}
\DeclareMathOperator{\add}{add}
\ifcsname R\endcsname\else\newcommand{\R}{\ensuremath{\mathbb{R}}}\fi

%% blocks world example
%\providecommand{\facton}[2]{\ensuremath{\text{on}(#1,#2)}\xspace}
%\providecommand{\factontable}[1]{\ensuremath{\text{on-table}(#1)}\xspace}
%\providecommand{\factclear}[1]{\ensuremath{\text{clear}(#1)}\xspace}
%\newcommand{\drawCube}[5]{
%    \draw[#4!90, fill=#4!50] (#1,#2,#3+#5) -- ++(0,#5,0) -- ++(#5,0,0) -- ++(0,-#5,0) -- cycle; % front
%    \draw[#4!90, fill=#4!50] (#1,#2+#5,#3) -- ++(0,0,#5) -- ++(#5,0,0) -- ++(0,0,-#5) -- cycle; % top
%    \draw[#4!90, fill=#4!50] (#1+#5,#2,#3) -- ++(0,#5,0) -- ++(0,0,#5) -- ++(0,-#5,0) -- cycle; % right
%}
\newcommand{\drawCube}[6]{
    \draw[#5!90, fill=#5!50] (#1,#2,#3+#6) -- ++(0,#6,0) -- ++(#6,0,0) -- ++(0,-#6,0) -- cycle; % front
    \draw[#5!90, fill=#5!50] (#1,#2+#6,#3) -- ++(0,0,#6) -- ++(#6,0,0) -- ++(0,0,-#6) -- cycle; % top
    \draw[#5!90, fill=#5!50] (#1+#6,#2,#3) -- ++(0,#6,0) -- ++(0,0,#6) -- ++(0,-#6,0) -- cycle; % right
    \node[align=center] at (#1+0.4*#6,#2+0.4*#6,#3+0.7*#6) {#4};
}
